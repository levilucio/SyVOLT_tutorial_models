\documentclass[conference]{IEEEtran}
\IEEEoverridecommandlockouts
% The preceding line is only needed to identify funding in the first footnote. If that is unneeded, please comment it out.
\usepackage{cite}
\usepackage{amsmath,amssymb,amsfonts}
\usepackage{algorithmic}
\usepackage{graphicx}
\usepackage{textcomp}
\def\BibTeX{{\rm B\kern-.05em{\sc i\kern-.025em b}\kern-.08em
    T\kern-.1667em\lower.7ex\hbox{E}\kern-.125emX}}
\begin{document}
 
\title{Build your own Rover with AutoFOCUS3\\
% {\footnotesize \textsuperscript{*}Note: Sub-titles are not captured in Xplore and
% should not be used}
% \thanks{Identify applicable funding agency here. If none, delete this.}
}

\author{\IEEEauthorblockN{The AF3 Team}
\IEEEauthorblockA{\textit{fortiss GmbH} \\ 
\textit{Guerickestrasse 25}\\
Munich, Germany \\
email address}
}

\maketitle 

\begin{abstract}
This is an awesome tutorial on AF3! 
\end{abstract}

\begin{IEEEkeywords}
Model-Based Software Engineering, IDE, embedded software
\end{IEEEkeywords}

\section{Presenters}
\begin{itemize}
  \item Levi L\'ucio
  \item who else?
\end{itemize}

\section{Proposed length}

6 hours

\section{Level of the Tutorial}

The tutorial will be split into 3 parts:

\begin{itemize}
  \item A gentle introduction to the modeling environment of AF3, followed by a
  first modelling exercise where the students will model a blinking light and
  make it run on a raspberry Pi.
  (2 hours)
  \item An exercise of intermediate difficulty where the software model of a
  rover controler is given and the students will model the blinker part (3
  hours)
  \item An advanced part where specific AF3 topics will be discussed (1 hour)
\end{itemize}

\section{Target Audience}

\section{Description of the Tutorial}
\subsection{What is AutoFOCUS3?}

\begin{itemize}
  \item Some history of AF3
  \begin{itemize}
    \item When did it start?
    \item Evolution (refer to published literature)
    \item Current state and release cycles
  \end{itemize}
  \item What is AUTOFocus aimed at? (developing software for embedded systems)
  \item Why is AUTOFocus a relevant tool? (covers full development cycle, formal
  verification is directly integrated)
  \item Comparison with competition (Papyrus, Enterprise Architect, MPS, \ldots)
\end{itemize}
\subsection{Part I: A simple blinker example}
\subsection{Part II: Integrating the blinker in the rover}
\subsection{Part III: Breakout sessions}

\section{Novelty of the Tutorial}

\section{Required Infrastructure} 

\end{document}
